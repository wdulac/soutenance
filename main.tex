\documentclass[aspectratio=169, usepdftitle=false, xcolor={dvipsnames}, 9pt,table]{beamer}

\usetheme[english, notoc, coloraccent=blue]{awesome}

\title{Méthodes pour l'évaluation de l'activité cyclonique tropicale en changement climatique}
\author[William]{William Dulac}
\subtitle{Soutenance de thèse}
\email{william.dulac@meteo.fr}
\institute{Centre National de Recherches Météorologiques}
\uni{Université Paul Sabatier -- Toulouse III} 
\location{Toulouse}
\date{20 Décembre 2023}

\supervisors{Julien Cattiaux, Fabrice Chauvin}
\reporters{Jean-Philippe Duvel, Sylvie Malardel}
\examinators{Jean-Pierre Chaboureau, Christophe Menkes,\par\hspace*{16ex}Caroline Muller}

\background{include/Bolaven_2023-10-11_2300Z.jpg}

%\usefonttheme[onlymath]{serif}
%\usefonttheme{professionalfonts}
%\usefonttheme{default}

%\setlength\abovecaptionskip{-1pt}

\addbibresource{references.bib}

\newcommand\blfootnote[1]{%
  \begingroup
  \renewcommand\thefootnote{}\footnote{#1}%
  \addtocounter{footnote}{-1}%
  \endgroup
}

\begin{document}

\maketitle

\section[Introduction]{Introduction : Cyclones tropicaux}

\makesecslide

\subsection{Définition}

%===================================================
\begin{frame}[t]
    \renewcommand*{\thefootnote}{\fnsymbol{footnote}}
    \frametitle{Qu'est-ce qu'un cyclone tropical (TC) ?}
    \framesubtitle{Définition et ordres de grandeur}
    \begin{center}
        \begin{minipage}{0.7\linewidth}
            \begin{definition}
                \footnotesize
                \centering
                Larges systèmes dépressionnaires et tourbillonnaires se développant sur océan dans les tropiques et caractérisés notamment par la présence d'un
            \textquote{œil} en leur centres
            \end{definition}
        \end{minipage}
    \end{center}
    %
    \begin{columns}
        \begin{column}{0.5\textwidth}
            \vspace{-3em}
            \begin{figure}
                \centering
                \includegraphics[width=\textwidth]{Figures/schema_cyclone.png}
                \caption{\textit{Inspiré d'une illustration de Frank Roux pour l'Encylopédie de l'Environnement}}
            \end{figure}
        \end{column}
        \begin{column}{0.5\textwidth}
           \setlength{\leftmargini}{2.5ex}
           \vspace{-2em}
           \begin{block}[Grandeurs caractéristiques] 
                \footnotesize
                \begin{itemize}
                    \item Fréquence annuelle moyenne : 84 TC par an \mbox{\parencite{schreck_impact_2014}}
                    \item Diamètre moyen de l'œil : 55 \sim~85 km \mbox{\parencite{weatherford_typhoon_1985}}
                    \item Diamètre\footnotemark~total moyen: 500 km \parencite{carrasco_influence_2014}
                    \item Vitesse de déplacement moyenne : 20 km/h
                    \item Vents soutenus : Entre 120 et 300 km/h
                \end{itemize}
            \end{block}
        \end{column}
    \end{columns}
    \footnotetext{Basé sur la mesure ROCI (\textit{Radius of Outermost Closed Isobar})}
    \renewcommand*{\thefootnote}{\arabic{footnote}}
    \setcounter{footnote}{0}
\end{frame}

%===================================================
\begin{frame}[c]
    \frametitle{Qu'est-ce qu'un cyclone tropical ?}
    \framesubtitle{Bassins d'activité et classification}
    \begin{columns}
        \begin{column}{0.65\textwidth}
            \begin{figure}[h]
                \centering
                \includegraphics[width=\textwidth]{Figures/Bassins_et_trajectoires_soutenance.png}
                \caption{\scriptsize Données issues de la basé de données de cyclones IBTrACS, version 4}
            \end{figure}
        \end{column}
        \begin{column}{0.35\textwidth}
            \vspace{-2em}
            \begin{table}
                \centering
                \footnotesize
                \caption{\centering \large Échelle de Saffir-Simpson}
                \begin{tabular}{l|c|c}
                     & Vent (m/s) & Pression (hPa) \\
                    \hline
                    TD & $<$ 16 & $>$ 1005 \\
                    TS & 16 -- 28 & 1005 -- 991 \\
                    Cat 1 & 29 -- 37 & 990 -- 976 \\
                    Cat 2 & 38 -- 43 & 975 -- 961 \\
                    Cat 3 & 44 -- 51 & 960 -- 946 \\
                    Cat 4 & 52 --  62 & 945 -- 926\\
                    Cat 5 & $\geq$ 63 & $\leq$ 925
                \end{tabular}
                \caption{\scriptsize Seuils de vents (sur 10 minutes) des catégories Saffir-Simpson (gauche) et seuils de pression équivalents ajustés
                selon \mbox{\cite{klotzbach_surface_2020}}.}
            \end{table}
            \begin{definition}
                \small Cyclone tropical = Ouragan = Typhon
            \end{definition}
        \end{column}
    \end{columns}
\end{frame}

%===================================================
\begin{frame}[c]
    \frametitle{Qu'est-ce qu'un cyclone tropical ?}
    \framesubtitle{Conditions de formations}
    \begin{figure}[h]
        \centering
        \includegraphics[width=0.8\textwidth]{Figures/diagramme_formation.png}
    \end{figure}
    %\vspace{1em}
    \footnotesize
    \begin{block}[Ingrédients de la cyclogénèse]
        \begin{columns}[t]
            \scriptsize
            \begin{column}{0.5\textwidth}
                \begin{itemize}
                   \item Température de surface de l'océan (SST) suffisamment élevée \mbox{\parencite{palmen_formation_1948}} sur au moins 60~m de profondeur 
                   \item Une distance de quelques centaines de kilomètres à l'équateur (Coriolis)
                   \item Perturbation initale apportant de la convergence en basses couches
                \end{itemize}
            \end{column}
            \begin{column}{0.5\textwidth}
               \begin{itemize}
                   %\setlength{\leftmargini}{1.5ex}
                   \item Absence de cisaillement vertical du vent\\\parencite{gray_global_1968}
                   \item Atmosphère humide en moyenne troposphère
               \end{itemize} 
            \end{column}
        \end{columns}
    \end{block}
    \blfootnote{Illustration adaptée d'une infographie AFP existante}
\end{frame}

%===================================================
\begin{frame}[c]
    \frametitle{Qu'est-ce qu'un cyclone tropical ?}
    \framesubtitle{Le phénomène météorologique le plus destructeur}
    \begin{definition}
        \footnotesize
        \begin{itemize}
            \item Plus de 400 000 morts et près de 300 000 blessés entre 1980 et 2009 \parencite{doocy_human_2013}.
            \item 20 millions de personnes rendues sans-abri \parencite{doocy_human_2013}.
            \item Coût économique moyen de plus de 20 milliards de dollars par cyclone (TC) aux États-Unis \parencite{smith_billiondollar_2020}.
        \end{itemize} 
    \end{definition}
    \begin{columns}
        \begin{column}{0.45\textwidth}
            \begin{figure}[h]
                \centering
                \includegraphics[width=0.97\textwidth]{Figures/idai_2019_sofala_province.jpg}
                \caption{Povince de Sofala au Mozambique après le passage du \mbox{cyclone} Idai, Mars 2019. \textit{Photo de l'Institut National
                de Gestion des Catastrophes du Mozambique (INGC).}}
            \end{figure}
        \end{column}
        \begin{column}{0.45\textwidth}
             \begin{figure}[h]
                 \centering
                 \includegraphics[width=\textwidth]{Figures/freddy_malawi_route.png}
                 \caption{Route endommagée au Malawi après le passage du cyclone Freddy, Mars 2023.\\\textit{Photo AP/SIPA/Thoko Chikondi}}
             \end{figure}
        \end{column}
    \end{columns}
\end{frame}


%=========================================================
\begin{frame}[c]
    \frametitle{Consensus sur l'évolution future de l'activité cyclonique}
    \begin{columns}
        \begin{column}{0.75\textwidth}
            \begin{figure}
                \centering
                \begin{tikzpicture}
                    \node<1->[anchor=south west, inner sep=0] (image) at (0,0)%
                        {\includegraphics[width=\textwidth]{Figures/Fig_5_Knutson_BAMS_revised_3_9_20-scaled_cropped.png}};
                    \begin{scope}[x={(image.south east)},y={(image.north west)}]
                        \draw<2->[rounded corners=0.5mm, red, fill, fill opacity=0.1, line width=0.5] (0.627, 0.018) rectangle (0.69, 0.055);
                        \draw<3->[rounded corners=0.5mm, green, fill, fill opacity=0.1, line width=0.5] (0.6955, 0.018) rectangle (0.749, 0.055);
                        \draw<4->[rounded corners=0.5mm, blue, fill, fill opacity=0.1, line width=0.5] (0.755, 0.018) rectangle (0.818, 0.055);
                        \draw<5->[rounded corners=0.5mm, purple, fill, fill opacity=0.1, line width=0.5] (0.825, 0.018) rectangle (0.898, 0.055);
                    \end{scope}
                \end{tikzpicture}
                \caption{\scriptsize Synthèse des projections de l'activité cyclonique pour un réchauffement de 2°C par rapport à 1986 -- 2005
                \mbox{\parencite{knutson_tropical_2020}}. Médianes et intervalles de confiance de respectivement 90~\% et 80~\% (gauche et droite).}
            \end{figure}
        \end{column}
        \begin{column}{0.25\textwidth}
           \footnotesize
           \setlength{\leftmargini}{2.5ex}
           \onslide<2->{
               \begin{block}[Résumé]
                   \scriptsize
                   \begin{itemize}
                        \item<2-> \textcolor{red}{Fréquence toutes catégories} : Baisse probable
                        \item<3-> \textcolor{green}{Fréquence des cyclones forts} : Hausse \underline{relative}
                        \item<4-> \textcolor{blue}{Intensité maximale} : Hausse fortement probable
                        \item<5-> \textcolor{purple}{Précipitations} : Hausse très fortement probable
                   \end{itemize}
               \end{block}
           }
           \onslide<6->{
               \begin{alertblock}
                    \scriptsize
                    Incertitudes conséquentes, y compris parfois sur le signe du changement
               \end{alertblock}
           }
        \end{column}
    \end{columns}
\end{frame}

\subsection[Méthodologies d'évaluation]{Comment sont réalisées ces projections ?}
\makesubsecslide

%=========================================================
\begin{frame}[c]
    \frametitle{Les modèles de climat}
    \framesubtitle{Simulation numérique de l'évolution de l'état de l'atmosphère}
    \begin{columns}
        \begin{column}{0.55\textwidth}
            \footnotesize
            \begin{definition} 
                Discrétisation des équations régissant l'évolution de l'état de \mbox{l'atmosphère} dans le temps et \alert{l'espace}\\
                $\longrightarrow$ ~L'échelle de la \alert{maille}
            \end{definition}
            \vspace{1em}%
            \onslide<2->{
                \begin{block}[Résolutions horizontales des modèles actuels \parencite{ipcc_annex_2021}]
                    \begin{itemize}
                        \item Hautes résolutions : 20 \sim~80 km
                        \item Basses résolutions : 100 \sim~250 km
                    \end{itemize}
                \end{block}
            }
            \vspace{1em}%
            \onslide<3->{
                \begin{alertblock}
                    On commence tout juste à pouvoir descendre à l'échelle nécessaire pour simuler des TC \underline{réalistes} aux échelles de temps
                    climatiques
                \end{alertblock}
            }
        \end{column}
        \onslide<1->{
            \begin{column}{0.45\textwidth}
                \begin{figure}[t]
                    \centering
                    \includegraphics[width=0.8\textwidth]{Figures/maillage_upscaled_x4_2.png}
                    \captionsetup{width=0.9\textwidth}
                    \caption{\scriptsize Illustration schématique du maillage tridimensionnelle d'un modèle de climat. Les couleurs représentent la température et les flèches
                    le vent.\\\textit{Illustration de \mbox{Laurent Fairhead/LMD/CNRS}}.}
                \end{figure}
            \end{column}
        }
    \end{columns} 
\end{frame}

%=========================================================
\begin{frame}
    \frametitle{Les réanalyses atmosphériques}
    \framesubtitle{Simulation numérique de l'état passé de l'atmosphère}
    \begin{columns}[c]
        \begin{column}{0.5\textwidth}
            \footnotesize
            \begin{block}
                \begin{itemize}
                \setlength\itemsep{3.5ex}
                    \item \alert{Reconstruction} spatiale (3D) et temporelle de l'état passé de l'atmosphère
                    \item Versions figées du modèle et du schéma d'assimilation\\
                        $\longrightarrow$ Jeu de données \alert{complet} et \alert{cohérent} dans le temps et l'espace
                    \item Souvent utilisées comme proxy d'observations \textit{in-situ}
                \end{itemize}
            \end{block}
        \end{column}
        \begin{column}{0.5\textwidth}
            \vspace{-2em}
            \begin{figure}
                \centering
                \includegraphics[width=\textwidth]{Figures/schema_reanalyse_top_custom.png}
                \animategraphics[autoplay,loop,width=\textwidth,poster=59]{13}{Figures/u850_octobre_2017/frame.00}{000}{123}
                \caption{\scriptsize Principe de fonctionnement de la réanalyse atmosphérique.\\Vent zonal à 850 hPa pour Octobre 2017 (ERA5).\\\textit{Adapté d'une illustration du CEPMMT (ECMWF)}}
            \end{figure} 
        \end{column}
    \end{columns} 
    %
    \begin{center}
        \begin{minipage}{12cm}
            \begin{definition}
                \centering
                \small
                Les réanalyses font le lien entre les sorties des modèles de climat et les observations
            \end{definition}
        \end{minipage}
    \end{center}
\end{frame}

%=========================================================
\begin{frame}[c]
    \frametitle{1\iere~approche : Détection et suivi objectif}
    \framesubtitle{Mesure directe dans les simulations haute résolution}
    \begin{columns}
        \begin{column}{0.66\textwidth}
            \footnotesize
            \begin{definition}
                \centering
                Algorithmes identifiant et reliant entre eux les points de grille qui satisfont une ou plusieurs conditions.
            \end{definition}
            \vspace{1em}
            \onslide<2->{
                \begin{block}
                    \centering
                    Statistiques descriptives sur le nombre et les propriétés des cyclones détectés dans des simulations en climats présent et futur
                \end{block}
            }
            \begin{columns}[t]
                \setlength{\leftmargini}{2.5ex}
                \begin{column}{0.45\textwidth}
                    \onslide<3->{
                        \begin{examples}[Intérêt principal]
                            \scriptsize
                            Mesure \alert{directe}
                            \tcblower
                            \scriptsize
                            \begin{itemize}
                                \item Capture le cycle de vie et la structure spatiale des cyclones
                                \item Permet une évaluation de tous les aspects de l'activité cyclonique
                            \end{itemize}
                        \end{examples}
                    }
                \end{column}
                \begin{column}{0.45\textwidth}
                    \onslide<4->{
                        \begin{alertblock}[Inconvénients] 
                            \scriptsize
                            \begin{itemize}
                                \item Nécessite des simulations à haute résolution (\alert{coûteuses} et encore trop \alert{peu nombreuses})
                                \vspace{2.9ex}
                                \item Influence du choix du traqueur \parencite{horn_tracking_2014,bourdin_intercomparison_2022}
                            \end{itemize}
                        \end{alertblock}
                    }
                \end{column}
            \end{columns}
            %\vspace{1em}
        \end{column}
        %
        \onslide<1->{
                \begin{column}{0.33\textwidth}
                \captionsetup{width=4cm}
                \vspace{-3em}
                \begin{figure}
                    \centering 
                    \animategraphics[autoplay,loop,width=4cm,poster=22]{5}{Figures/Ophelia/vo850/ophelia-vo850-}{0}{27}
                    \caption{\href{run:./ophelia-vo850.gif}{Animation} du cyclone Ophelia (2017) sur fond de \alert{vorticité relative à 850 hPa} (ERA5)}
                \end{figure}
                %
                \begin{figure}
                    \centering
                    \animategraphics[autoplay,loop,width=4cm,poster=22]{5}{Figures/Ophelia/msl/ophelia-msl-}{0}{27}
                    \caption{\href{run:./ophelia-msl.gif}{Animation} du cyclone Ophelia (2017) sur fond de \alert{pression au niveau de la mer} (ERA5)}
                \end{figure}
            \end{column}
        }
    \end{columns}
\end{frame}

%=========================================================
\begin{frame}[t]
    \frametitle{2\ieme~ approche : Indices de cyclogénèse}
    \framesubtitle{Inférence de l'activité à travers l'environnement de grande échelle}
    \begin{columns}[t]
        \begin{column}{0.66\textwidth}
            \begin{definition}
                \scriptsize
                %\centering
                \underline{Fréquence d'occurrence} des TC en un point de grille proportionnelle à une \mbox{combinaison} de variables \alert{thermiques} et
                \mbox{\alert{dymamiques}} de grande échelle \parencite{gray_tropical_1975}.
            \end{definition}
            \onslide<2->{
                \vspace{1em}
                \begin{block}
                    \scriptsize
                    \centering
                    %Relations établies en climat présent pour application à des simulations en climat futur.
                    Relations caractérisant la climatologie spatio-temporelle présente, et appliquées à des simulations en climat futur.
                \end{block}
            }
            \begin{columns}[t]
                \setlength{\leftmargini}{2ex}
                \begin{column}{0.45\textwidth}
                    \onslide<3->{
                        \footnotesize
                        \begin{examples}[Intérêt principal]
                            \scriptsize
                            Applicable à des modèles basse résolution\\(plus nombreux \Rightarrow Meilleure estimation de l'incertitude)
                        \end{examples}
                    }
                \end{column}
                \begin{column}{0.45\textwidth}
                    \onslide<4->{
                        \footnotesize
                        \begin{alertblock}[Inconvénients]
                            \scriptsize
                            \begin{itemize}
                                \item Mesure \alert{indirecte}
                                \item Incertitudes sur le choix des variables
                                \item Hypothèse de stationnarité 
                            \end{itemize}
                        \end{alertblock}
                    }
                \end{column}
            \end{columns}
        \end{column}
        \begin{column}{0.33\textwidth}
            \vspace{-6.5em}
            \begin{figure}
                \centering
                \includegraphics[width=0.88\textwidth]{Figures/acgi_seasonal_turbo.png}
                %\includegraphics[width=\textwidth]{Figures/acgi_mean_IBTrACS_season.png}
                %\caption{\scriptsize Moyennes saisonnières de trois indices moyennés entre eux (GPI, TCS et CYGP), et densité observée de cyclogénèse entre 1980 et 2019
                %(IBTrACS).}
            \end{figure}        
        \end{column}
    \end{columns}
\end{frame}

%=========================================================
\subsection{Problématique}
\begin{frame}[t]
    \frametitle{Problématique}
    \framesubtitle{Contexte général}
    \vspace{-1em}
    \begin{columns}[b]
        \begin{column}{0.5\textwidth}
            \begin{figure}[h]
                \centering
                \includegraphics[width=\textwidth]{Figures/density_fitted_title2.png}
                \caption{\scriptsize Changement dans la densité de cyclogénèse (par 50 ans) par mesure \mbox{\alert{directe}} sur la période 2031 -- 2080 (RCP
                8.5) par rapport à 1965 -- 2013 \mbox{\parencite{chauvin_future_2020}}}
            \end{figure}
        \end{column}
        \begin{column}{0.5\textwidth}
            \begin{figure}[h]
                \centering
                \includegraphics[width=\textwidth]{Figures/gpi_fitted_2_good_title2.png}
                \caption{\scriptsize Changement dans la densityé de cyclogénèse (par 50 ans) par mesure \mbox{\alert{indirecte}} (GPI) sur la période 2031 --
                2080 (RCP 8.5) par rapport à 1965 -- 2013 \mbox{\parencite{chauvin_future_2020}}}
            \end{figure}
        \end{column}
    \end{columns}
    %
    \begin{center}
        \vspace{-1ex}
        \begin{minipage}{7cm}
            \begin{alertblock}
                \centering
                \footnotesize
                \underline{\alert{Divergence}} des projections issues des deux approches
                %Les projections \underline{\alert{divergent}}
            \end{alertblock}
        \end{minipage}
    \end{center}
\end{frame}

%=========================================================
\begin{frame}[t]
   \frametitle{Problématique}
   \framesubtitle{Recherches menées}
   \vspace{-2em}
   \begin{columns}[t]
       \begin{column}{0.475\textwidth}
           \onslide<1->{
               \begin{center}
                   \large
                   \textbf{Détection et suivi} 
               \end{center}
           }
           \vspace{0.8em}
           \footnotesize
           \onslide<2->{
               \begin{alertblock}[Verrous identifiés]
                   \setlength{\leftmargini}{2.5ex}
                   \begin{itemize}
                       \item Influence du choix du traqueur dans les projections \parencite{horn_tracking_2014} 
                       \item Qu'est-ce que détecte réellement un traqueur ? (\textit{Hurricane Type Vortices}, HTV, \cite{bengtsson_hurricanetype_1995})
                   \end{itemize}
               \end{alertblock}
           }
           \vspace{1em}
           \onslide<3->{
               \begin{examples}[Axe de recherche]
                   Caractériser en détails le schéma de détection du CNRM pour identifier des pistes de perfectionnement
               \end{examples}
           }
           \vspace{1em}
           \onslide<4->{
               $\longrightarrow$ Développement de métriques d'évaluation des performances d'un schéma de détection
           }
       \end{column}
       %
       \begin{column}{0.475\textwidth}
           \onslide<1->{
               \begin{center}
                   \large
                   \textbf{Indices de cyclogénèse} 
               \end{center}
           }
           \vspace{0.5em}
           \footnotesize
           \onslide<5->{
               \begin{alertblock}[Verrous identifiés]
                   \setlength{\leftmargini}{2.5ex}
                   \begin{itemize}
                       \item Tendances projetées à l'opposé de l'approche directe \parencite{camargo_testing_2014}
                       \item Mauvaise représentation de la variabilité interannuelle \parencite{menkes_comparison_2012,cavicchia_tropical_2023}
                   \end{itemize}
               \end{alertblock}
           }
           \vspace{1em}
           \onslide<6->{
               \begin{examples}[Hypothèse de travail]
                   Améliorer la variabilité historique pour avoir une meilleure confiance dans les projections futures
               \end{examples}
           }
           \vspace{1em}
           \onslide<7->{
               $\longrightarrow$ Amélioration de la représentation de la variabilité interannuelle et des tendances historiques par les indices
           }
       \end{column}
   \end{columns}
\end{frame}



%=========================================================
\section[Détection et suivi]{Détection et suivi : Application du traqueur CNRM à la réanalyse ERA5}
\begin{frame}[c]
    \frametitle{Plan}
    \addtocounter{framesinsection}{-1}
    \tableofcontents[currentsection,hideallsubsections,sections={1-2}]
    \vspace{-3em}
    \tableofcontents[hideallsubsections,sections={3-}]
\end{frame}

%\begin{frame}[c]
%    \frametitle{Plan}
%    \addtocounter{framesinsection}{-1}
%    \tableofcontents[currentsection,hideothersubsections]
%\end{frame}

\makesecslide

%=========================================================
\subsection*{Motivations}
%\begin{frame}[t]
%    \frametitle{Introduction}
%    \framesubtitle{Intérêts de la réanalyse ERA5}
%    \onslide<1->{
%        \begin{center}
%            \begin{minipage}{10cm}
%                \small
%                \begin{definition}[Réanalyse ERA5 \parencite{hersbach_era5_2020}]
%                    \footnotesize
%                    \begin{itemize}
%                        \item Dernière réanalyse en date du CEPMMT
%                        \item Résolution horizontale $\sim$ 0.28° (25 km) sur tout le globe
%                        \item De 1940 à aujourd'hui (en cours)
%                    \end{itemize}
%                \end{definition}
%            \end{minipage}
%        \end{center}
%    }
%    %
%    \vspace{1.5em}
%    %
%    \begin{columns}[t]
%        \begin{column}{0.5\textwidth}
%            \onslide<2->{
%                \small
%                \begin{block}[Contexte de la thèse]
%                    \footnotesize
%                    Meilleure approximation de l'état \underline{historique} de l'atmosphère
%                    \tcblower
%                    \begin{itemize}
%                        \footnotesize
%                        \item La plupart des TC observés sont présents dedans
%                        \item Permet la validation du schéma de détection
%                    \end{itemize}
%                \end{block}
%            }
%        \end{column}
%        \begin{column}{0.5\textwidth}
%            \onslide<3->{
%                \small
%                \begin{block}[Intérêt scientifique]
%                    \begin{itemize}
%                        \footnotesize
%                        \item Produit de référence utilisé pour combler le manque entre les observations et les sorties de modèle 
%                        \item Beaucoup d'attentes pour la représentation des TC en raison de sa haute résolution
%                    \end{itemize}
%                \end{block}
%            }
%        \end{column}
%    \end{columns}
%    % Slide d'introduction pour justifier le fait de travailler sur ERA5
%    % \begin{itemize}
%    %     \item Dans le contexte de la thèse : validation de l'outil de tracking
%    %     \item Intérêt pour la communauté scientifique : \textquote{Nouveau} produit de référence, haute résolution, beaucoup d'espoir pour la représentation des
%    %         TC.
%    % \end{itemize}
%\end{frame}

\begin{frame}[t]
    \frametitle{Introduction}
    \framesubtitle{Intérêts de la réanalyse ERA5}
    \small
    \begin{definition}[Réanalyse ERA5 \parencite{hersbach_era5_2020}]
        \footnotesize
        \begin{itemize}
            \item Dernière réanalyse en date du CEPMMT
            \item Résolution horizontale $\sim$ 0.28° (environ \alert{30 km}) sur tout le globe : Beaucoup d'attentes pour les cyclones
            \item De 1940 à aujourd'hui (en cours)
            \item Produit de référence pour les années à venir
        \end{itemize}
    \end{definition}
    %
    \vspace{1.5em}
    \small
    \begin{block}[Intérêt de l'utilisation de la réanalyse ERA5]
        \footnotesize
        Meilleure approximation de l'état \underline{historique} de l'atmosphère
        \tcblower
        \begin{itemize}
            \footnotesize
            \item La plupart des TC observés sont présents dedans
            \item Permet la validation du schéma de détection
        \end{itemize}
    \end{block}
\end{frame}
 
%=========================================================
\subsection{Données et outils}
\begin{frame}[t]
    \frametitle{Schéma de détection du CNRM}
    \footnotesize
    \begin{block}[Étapes de détection \parencite{chauvin_response_2006}] 
        \scriptsize
        \begin{enumerate}
            \item<1-> Minimum local de pression avec \underline{vorticité} supérieure à un \textcolor[HTML]{ff0011}{seuil} (\textbf{VOR})
            \item<2-> Estimation de la taille du \textcolor{green}{système} \onslide<3->{et de son \textcolor{blue}{environnement}}
                \onslide<4->{
                    \begin{itemize}
                        \scriptsize
                        \item Seuil de vent à 10~m (\textbf{RES})
                        \item Anomalie de température cumulée à 700~hPa, 500~hPa et 300~hPa (\textbf{TANOM})
                        \item Profil vertical de température entre 300~hPa et 850~hPa (\textbf{PT})
                        \item Profil vertical de vent entre 300~hPa et 850~hPa (\textbf{PW})
                    \end{itemize}
                }
            \item<5-> Suivi du système (appariement des points)
            \item<6-> \textcolor{violet}{Relaxation} des paramètres pour \underline{compléter} en amont et en aval (\textbf{REL})
        \end{enumerate}
    \end{block}
    %\vspace{1em}
    \begin{figure}
        \centering
        %\includegraphics<1>[height=2.8cm]{Figures/fonctionnement_tracker/step0.png}%
        \includegraphics<1>[height=2.8cm]{Figures/fonctionnement_tracker/step1.png}%
        \includegraphics<2>[height=2.8cm]{Figures/fonctionnement_tracker/step2.png}%
        \includegraphics<3>[height=2.8cm]{Figures/fonctionnement_tracker/step3.png}%
        \includegraphics<4>[height=2.8cm]{Figures/fonctionnement_tracker/step4.png}%
        \includegraphics<5>[height=2.8cm]{Figures/fonctionnement_tracker/step5.png}%
        \includegraphics<6>[height=2.8cm]{Figures/fonctionnement_tracker/step6.png}%
    \end{figure}
\end{frame}

%=========================================================
\begin{frame}[t]
    \frametitle{Application à la réanalyse ERA5}
    \framesubtitle{Données et outils}
    \begin{columns}[t]
        \begin{column}{0.6\textwidth}
            \vspace{-1.5em}
            \onslide<1->{
                \scriptsize
                \begin{examples}[Jeux de données]
                    \scriptsize
                    \setlength{\leftmargini}{2.5ex}
                    \begin{itemize}
                        \item Base de données d'observations IBTrACS (version 4) 1981 -- 2019 \parencite{knapp_international_2010}
                        \item Réanalyse ERA5 à 0.25° 1981 -- 2019 \parencite{hersbach_era5_2020}
                    \end{itemize}
                    $\longrightarrow$ Jeu de trajectoires de \alert{référence}
                    \vspace{-1em}
                \end{examples}
            }
            \vspace{1em}
            \onslide<2->{
                \scriptsize
                \begin{block}[Appariement des trajectoires]
                    \scriptsize
                    Basé sur la correspondance \alert{spatio-temporelle}
                    \setlength{\leftmargini}{4.5ex}
                    \begin{enumerate}
                        %\item Recherche des candidats IBTrACS par correspondance \alert{temporelle}
                        \item \alert{Score} basé sur la quantité de points en commun à moins de 300 km
                        \item Sélection de la \alert{paire de trajectoires} avec le meilleur score
                    \end{enumerate}
                    \textcolor{red}{Une seule échéance commune suffit !}
                    \vspace{-1em}
                \end{block}
            }
            \vspace{1em}
            \onslide<3->{
                \scriptsize
                \begin{definition}[Optimisation du schéma de détection pour ERA5 \parencite{dulac_assessing_2023}]
                    \setlength{\leftmargini}{2.5ex}
                    \begin{enumerate}
                        \item Calcul de l'efficacité de détection (POD) et du taux de fausses détections (FAR).
                        \item Seuils de détection du traqueur qui \underline{optimisent} le POD et le FAR sur ERA5
                    \end{enumerate}
                    $\longrightarrow$ Jeu de trajectoires \alert{optimisé}
                    %\textbf{VOR} = 15$\cdot$10$^{\text{-5}}$ s$^{\text{-1}}$ ; \textbf{RES} = 5 m.s$^{\text{-1}}$ ; \textbf{TANOM} = 1~K ;
                    %\textbf{PT} = -1~K ; \textbf{PW} = 5 m.s$^{\text{-1}}$ ; \textbf{REL} = 25$\cdot$10$^{\text{-5}}$ s$^{\text{-1}}$
                    \vspace{-1em}
                \end{definition}
            }
        \end{column}
        \onslide<2->{
            \begin{column}{0.4\textwidth}
                \vspace{-3.5em}
                \begin{figure}
                    \captionsetup{width=5cm}
                    \centering
                    \includegraphics[width=5cm]{Figures/katrina_era5.png}
                    \caption{\scriptsize Ouragan Katrina (2005) observée (rouge) et détectée dans ERA5 (bleu), résultant de l'appariement des trajectoires.}
                \end{figure}
            \end{column}
        }
    \end{columns} 
\end{frame}


%=========================================================
\subsection*{Performances du traqueur}
\begin{frame}[t]
    \frametitle{Performances du traqueur sur ERA5}
    \footnotesize
    %\vspace{-1.5em}
    \begin{columns}
        \begin{column}{0.56\textwidth}
        \footnotesize
        %\vspace{1.5em}
        \begin{figure}
            \centering
            \includegraphics[height=4cm]{Figures/far_pod_avant_apres.png}
        \end{figure}
        \end{column}
        \begin{column}{0.4\linewidth}
        \footnotesize
        \vspace{-1em}
        \begin{block}        
            Valeurs seuils \alert{optimisées} + filtre des systèmes de moyennes latitudes \parencite{hart_cyclone_2003}
            \tcblower
            \setlength{\leftmargini}{2.5ex}
            \alert{Majorité} des systèmes détectés dans ERA5 :
            \begin{itemize}
                \item POD global : 67,4~\% (+11,3 points)
                \item FAR global : 23,5~\% (-0,5 points)
            \end{itemize}
            %$\longrightarrow$ Appariement permet d'étudier \underline{uniquement} les \alert{\mbox{systèmes} identifiés} dans IBTrACS
        \end{block}
        \end{column}
    \end{columns}
    \vspace{1em}
    %
    %\pause
    %
    \begin{examples}[Intercomparaison de traqueurs sur ERA5]    
        Participation du traqueur CNRM à une \alert{intercomparaison} de 4 traqueurs appliquée sur ERA5 \parencite{bourdin_intercomparison_2022}
        \tcblower
        $\longrightarrow$ Performances similaires à d'autres traqueurs de la littérature
    \end{examples}
\end{frame}


%=========================================================
\subsection[Représentation des cyclones dans ERA5]{Représentation des cyclones dans ERA5}
\begin{frame}
    \frametitle{Relation vent-pression dans ERA5}
    \begin{columns}
        \begin{column}{0.5\textwidth}
            \begin{figure}
                \centering
                \includegraphics[width=\textwidth]{Figures/ERA5_PV_myVTU.png}
            \end{figure}
        \end{column}
        \begin{column}{0.5\textwidth}
            \footnotesize
            \setlength{\leftmargini}{3.5ex}
            \vspace{1.25ex}
            \begin{examples}[Méthodologie] 
                \begin{itemize}
                    \item Ensemble des échéances partagées pour chaque paire de systèmes\\(58 920 points dans chacun des jeux de données)
                    \item Échelle globale (tous bassins sauf Nord Indien)
                    \item Fit loi de puissance $V = a\Delta P^b$ \parencite{atkinson_tropical_1977}
                \end{itemize}
            \end{examples}
            \vspace{2em}
            \begin{block}
                \begin{itemize}
                    \item \alert{Forte sous-estimation} des vents maximum dans ERA5
                    \item Distribution des minimum de pression \underline{mieux représentée} (Cat 4)
                    \item Correspondance imparfaite des catégories ERA5 / IBTrACS
                \end{itemize}
            \end{block}
        \end{column}
    \end{columns} 
\end{frame}


%=========================================================
\begin{frame}[t]
    \frametitle{Décalage temporel dans le cycle de vie}
    \begin{figure}
        \centering
        \includegraphics[width=0.8\textwidth]{Figures/lag_max_intensity_myVTU.png}
    \end{figure}
    \begin{columns}
        \footnotesize
        \setlength{\leftmargini}{3ex}
        \begin{column}{0.5\textwidth}
            \begin{examples}
                \begin{itemize}
                    \item Délais (en heures) entre le maximum d'intensité des paires ERA5/IBTrACS par classe d'intensité (IBTrACS)
                    \item Délai > 0 indique que le TC dans ERA5 est \underline{en retard}
                \end{itemize}
            \end{examples}
        \end{column}
        \begin{column}{0.5\textwidth}
            \begin{block} 
                \begin{itemize}
                    \item Délai moyen toutes catégories de \alert{+13 heures} (significatif)
                    \item Délai \alert{augmente} avec la classe d'intensité\\(45 heures pour Cat 5)
                \end{itemize}
            \end{block}
        \end{column}
    \end{columns}
\end{frame}


%=========================================================
\begin{frame}[c]
    \frametitle{Représentation des cyclones dans ERA5}
    \framesubtitle{Conclusion}
    %Slide de conclusion sur cette partie : Pourquoi ces résultats sont importants :
    %\begin{itemize}
    %    \item Attention quand on veut aller voir la tête d'un TC observé dans ERA5 (vent sous-estimé + pas forcément même classe d'intensité)
    %    \item C'est également fortement déconseillé d'aller directement extraire les champs ERA5 en lieu et place du max d'intensité d'un TC observé (à cause du
    %        délai)
    %\end{itemize}
    \begin{block}[Messages clefs \parencite{dulac_assessing_2023}]
        \small
        \begin{enumerate}
            \setlength\itemsep{1em}
            \item<1-> \alert{Bonnes performances} du traqueur (FAR et POD), grâce à l'optimisation et à l'utilisation d'un filtre de systèmes de
                moyennes latitudes adapté
            \item<2-> Forte \alert{sous-estimation} des vents associés aux TC, pression minimale mieux représentée
            \item<3-> \alert{Retard} conséquent dans l'intensification (45 heures pour Cat 5)
        \end{enumerate}
        \vspace{1em}
        \onslide<4->{\hskip1em $\longrightarrow$ \textcolor{red}{Attention} pour les études de cas de TC historiques!}
    \end{block}
    \onslide<5->{
        \large 
        \vspace{\baselineskip}
        \begin{center}
            POD amélioré par l'optimisation, qu'en est-il de la \alert{\textit{ressemblance}} ?
            %Vers des métriques d'évaluation de la \alert{qualité} des trajectoires ?
        \end{center}
    }
\end{frame}

%=========================================================
\subsection[Similarité des trajectoires]{Métriques de similarité des trajectoires}
\makesubsecslide
\begin{frame}[c]
    \frametitle{Similarité des trajectoires}
    \framesubtitle{Motivations}
    \small
    \vspace{1em}
    \begin{definition}
        \begin{itemize}
            \item Métriques de performances du traqueur \alert{intégrées} sur les trajectoires (par rapport au FAR et POD)
            %\item Proportion non-nulle des composantes ERA5 de chaque paire de trajectoires qui se trouve à plus de \mbox{300 km} de la composante IBTrACS
            \item Appariement $<$ 100~\% : Trajectoires détectées trop courtes \alert{et/ou} située à plus de 300 km d'IBTrACS
        \end{itemize}
    \end{definition}
    %\vspace{1em}
    \begin{figure}
        \centering
        \includegraphics[height=4cm]{Figures/coverage_ratio.png}
        \captionsetup{width=0.75\textwidth}
        \caption{\scriptsize Statistiques réalisées sur les 2 244 paires de trajectoires de \cite{dulac_assessing_2023} dont l'intensité de la composante IBTrACS est
        classifiable (moins de 75\% de valeurs manquantes) et sur les 5 bassins géographiques (NInd exclu).}
    \end{figure}
\end{frame}

%=========================================================
\begin{frame}[c]
    \frametitle{Similarité des trajectoires}
    \framesubtitle{Principe de fonctionnement}
    \begin{columns}[t]
        \begin{column}{0.4\textwidth}
           \onslide<1->{ 
               \vspace{-1em}
               \footnotesize
               \begin{definition}
                   \scriptsize
                   \centering
                   Décomposition d'une trajectoire en séquence de vecteurs de déplacement relatifs
               \end{definition}
           }
           %
           \onslide<4->{
               \footnotesize
               \begin{examples}[Similarité angulaire $S_\theta$]
                   \scriptsize
                   Pour deux vecteurs au temps $t$ issus de deux trajectoires différentes :
                   %
                   \onslide<5->{
                       \setlength{\belowdisplayskip}{-2mm}
                       \begin{align*}
                           \cos \theta &= \frac{\mathbf{c}_i \cdot \mathbf{q}_i}{\lVert \mathbf{c}_i \rVert \lVert \mathbf{q}_i \rVert} \in [-1; 1] \\
                           S_{\theta} &= 1 - \frac{\theta}{\pi} \in [0; 1]
                       \end{align*}
                   }
               \end{examples}
           }
           %
           \onslide<6->{
           \footnotesize
           \begin{block}
               \scriptsize
               \setlength{\leftmargini}{2.5ex}
               \begin{itemize}
                   \item<6-> \alert{Intersection} $C \cap Q$ : Points où les deux trajectoires co-existent aux mêmes temps $t$
                   \item<7-> \textcolor{green}{Différence symmétrique} $C \ominus Q$ : Points où une seule trajectoire est définie
                   \item<8-> Union $C \cup Q$ : Ensemble des points des deux trajectoires
               \end{itemize}
           \end{block}
       }
        \end{column}
        \begin{column}{0.6\textwidth}
            \vspace{-3.5em}
            \begin{figure}
                \centering
                \includegraphics<1>[width=\textwidth]{Figures/schema_vec_sequence/step0.png}%
                \includegraphics<2>[width=\textwidth]{Figures/schema_vec_sequence/step1.png}%
                \includegraphics<3>[width=\textwidth]{Figures/schema_vec_sequence/step2.png}%
                \includegraphics<4>[width=\textwidth]{Figures/schema_vec_sequence/step3.png}%
                \includegraphics<5>[width=\textwidth]{Figures/schema_vec_sequence/step4.png}%
                \includegraphics<6->[width=\textwidth]{Figures/schema_vec_sequence/step5.png}%
            \end{figure} 
        \end{column}
    \end{columns} 
\end{frame}

%=========================================================
\begin{frame}[t]
    \frametitle{Métriques de similarité}
    %\vskip0ptplus1filll\relax
    \begin{columns}[t]
        \begin{column}{0.66\textwidth}
            \footnotesize
            \vspace{-1.5em}
            \begin{examples}[Métriques de similarité angulaire moyenne ({MAS, \textit{Mean Angular Similarity}})]
                \begin{enumerate}[(a)]
                    \item<1-> $\text{MAS}^\cap = \bar{S_\theta}^\cap = \frac{1}{\alert{N_\cap}} \sum_{i=1}^{N_\cap} S_\theta (\mathbf{c}_i, \mathbf{q}_i)$
                    \item<3-> $\text{MAS}^\cup = \bar{S_\theta}^\cup = \frac{1}{\underbrace{\textcolor{blue}{N_\cap} + \textcolor{red}{N_\ominus}}_{=
                        N_\cup}} \sum_{i=1}^{N_\cap} S_\theta (\mathbf{c}_i, \mathbf{q}_i)$
                    \item<4-> $\text{wMAS} = \frac{ \alert{w_1}}{ \alert{w_1 N_\cap} + \textcolor{red}{w_2 N_\ominus}} \sum_{i=1}^{N_\cap} S_{\theta}(\mathbf{c}_i, \mathbf{q}_i) \quad\quad 
                    \begin{cases}
                        \alert{w_1} = 1 + \bar{S_\theta}^\cap \\
                        \textcolor{red}{w_2} = 2 - \bar{S_\theta}^\cap \\
                    \end{cases}$
                \end{enumerate}
            \end{examples}
        \end{column}
        \begin{column}{0.33\textwidth}
            \footnotesize
            \vspace{-7em}
            \begin{figure}
                \centering
                \includegraphics<1-2>[width=0.9\textwidth]{Figures/exemple_similarite/step1.png}%
                \includegraphics<3>[width=0.9\textwidth]{Figures/exemple_similarite/step2.png}%
                \includegraphics<4->[width=0.9\textwidth]{Figures/exemple_similarite/step3.png}%
            \end{figure}
        \end{column}
    \end{columns}
    \vspace{-0.5em}
    \begin{figure}
        \centering
        \includegraphics<2>[height=3.5cm]{Figures/MAS_vs_NDIFF/step1.png}%
        \includegraphics<3>[height=3.5cm]{Figures/MAS_vs_NDIFF/step2.png}%
        \includegraphics<4->[height=3.5cm]{Figures/MAS_vs_NDIFF/step3.png}%
    \end{figure}
\end{frame}

%=========================================================
\begin{frame}[c]
    \frametitle{Application des métriques}
    \framesubtitle{Optimisation des paramètres du traqueur : Quel impact ?}
    \small
    \begin{definition}
        \centering
        Calcul de la différence de similarité dans les paires ERA5 / IBTrACS avant et après l'optimisation des paramètres
    \end{definition}
    \vspace{1em}
           %\begin{block}[Jeux de trajectoires]
           %    1 649 \alert{triplets} de trajectoires :
           %    \begin{itemize}
           %         \item IBTrACS v4
           %         \item Trajectoires ERA5 utilisées dans \cite{dulac_assessing_2023} (Optimisées)
           %         \item Trajectoires ERA5 \alert{avant} optimisation des paramètres :
           %             \begin{itemize}
           %                 \footnotesize
           %                 \item \textbf{VOR} = 5$\cdot$10$^{\text{-5}}$ s$^{\text{-1}}$ (resp. 15$\cdot$10$^{\text{-5}}$ s$^{\text{-1}}$)
           %                 \item \textbf{RES} = 15 m.s$^{\text{-1}}$ (resp. 5m~.s$^{\text{-1}}$)
           %                 \item \textbf{PT} = -2~K (resp. -1~K)
           %                 \item Paramètre de relaxation \alert{identique} (25$\cdot$10$^{\text{-5}}$ s$^{\text{-1}}$)
           %             \end{itemize}
           %    \end{itemize}
           %    $\longrightarrow$ \alert{Deux trajectoires} ERA5 pour chaque trajectoire IBTrACS
           %\end{block}
    \vspace{-1em} 
    \begin{figure}
        \centering
        \includegraphics[height=5cm]{Figures/schema_application_similarite.png}
    \end{figure} 
\end{frame}

%=========================================================
\begin{frame}[c]
    \frametitle{Changements dans les similarités}
    \framesubtitle{Résultats}
    \begin{figure}
        \centering
        \includegraphics[height=3.8cm]{Figures/delta_MAS.png}
    \end{figure}
    \begin{columns}
        \footnotesize
        \begin{column}{0.5\textwidth}
            \onslide<1->{
                \begin{block}
                    \begin{itemize}
                        \item $\overline{\Delta N_\ominus} =$ -0,8 échéances
                        \item $\sim$ 73~\% de la variance expliquée par $\Delta N_\ominus$
                    \end{itemize}
                    \tcblower
                    $\longrightarrow$ Amélioration de la similarité par \alert{réduction} da la différence de temporalité
                \end{block}
            }
        \end{column}
        \begin{column}{0.5\textwidth}
            \onslide<2->{
                \begin{alertblock}[Lorsque {$\Delta N_\ominus = 0$} \underline{et} {$\Delta N_\cap = 0$}]
                    À temporalité \alert{inchangée} ($n =$ 691) :
                    \begin{enumerate}[(a)]
                        \item $\overline{\Delta \text{MAS}^\cap} =$ +0,57 p.p
                        \item $\overline{\Delta \text{MAS}^\cup} =$ +0,34 p.p
                        \item $\overline{\Delta \text{wMAS}} =$ +0,54 p.p
                    \end{enumerate}
                \end{alertblock}
            }
        \end{column}
    \end{columns}
\end{frame}

%=========================================================
\begin{frame}
    \frametitle{Métriques de similarité des trajectoires}
    \framesubtitle{Synthèse}
    \begin{block}[Messages clefs]
        \small
        \begin{enumerate}
            \setlength\itemsep{1em}
            \item<1-> Nouvelles métriques d'évaluation des performances du traqueur basées sur la similarité angulaire des trajectoires détectées\\
                $\longrightarrow$ Combine une information sur la position et sur la durée des trajectoires détectées
            \item<2-> Distinction des métriques selon leur comptabilisation des échéances non communes\\
                $\longrightarrow$ Peuvent être utilisées de manière complémentaires
            \item<3-> Scores de similarité élévés sur les trajectoires détectées\\
                $\longrightarrow$ Amélioration de la similarité suite à l'optimisation du traqueur (meilleure temporalité)
        \end{enumerate}
    \end{block}
\end{frame}

%  \begin{frame}
%      \frametitle{Détection et suivi des TC dans ERA5}
%      \framesubtitle{Synthèse}
%      \begin{block}[Messages clefs]
%          \small
%          \begin{enumerate}
%              \setlength\itemsep{1em}
%              \item<1-> Utilisation de la réanalyse ERA5 pour caractériser le schéma de détection du CNRM
%              \item<2-> Appariement des trajectoires détectées avec les trajectoires observées
%              \item<3-> Métrique FAR et POD : Détection binaire
%              \item<4-> Métrique de similarité des trajectoires : Combine une information sur la position et sur la durée
%          \end{enumerate}
%      \end{block}
%  \end{frame}
 
%=========================================================
\section[Indices de cyclogénèse]{Indices de cyclogénèse : Vers une meilleure représentation de la variabilité historique}
%begin{frame}[c]
%    \frametitle{Plan}
%    \addtocounter{framesinsection}{-1}
%    \tableofcontents[currentsection,hideallsubsections,sections={1-3}]
%    \vspace{-3em}
%    \tableofcontents[hideallsubsections,sections={4-}]
%\end{frame}
%
\begin{frame}[c]
    \frametitle{Plan}
    \addtocounter{framesinsection}{-1}
    \tableofcontents[currentsection,hideothersubsections]
\end{frame}

%\makesecslide

%=========================================================
\subsection{Introduction}
\begin{frame}[t]
    \frametitle{Introduction}
    \framesubtitle{Les indices de cyclogénèse}
    \footnotesize
    \begin{columns}
        \begin{column}{0.7\textwidth}
            \vspace{-3em}
            \begin{definition}[Définition]
                \alert{Relation} mathématique reliant la fréquence d'occurrence des TC (cyclogénèses) avec l'environnement de grande échelle, assimilables à une densité
                surfaçique :
                \begin{align*}
                    N_{\text{TC}} &= \iint\limits_{D} \underbrace{\rho(\phi, \lambda)}_{\text{Indice}} \; d\phi \, d\lambda \\
                    \rho &= \underbrace{\vphantom{X_{\text{Dyn}}}\textcolor{green}{A}}_{\textcolor{green}{\text{Calibration}}} \times
                    \underbrace{\vphantom{X_{\text{Dyn}}}\textcolor{red}{X_{\text{Th}}}}_{\textcolor{red}{\text{Thermique}}} \times
                    \underbrace{\textcolor{violet}{X_{\text{Dyn}}}}_{\textcolor{violet}{\text{Dynamique}}}
                \end{align*}
                Constitués d'une composante \textcolor{red}{thermique}, d'une composante \textcolor{violet}{dynamique} et sont \textcolor{green}{calibrés} pour simuler la
                quantité moyenne souhaitée 
            \end{definition}
            \vspace{2em}
            \uncover<2->{
                \begin{alertblock}
                    \centering
                    Les indices sont initialement conçus pour reproduire la \underline{climatologie} spatiale et saisonnière de l'activité cyclonique
                \end{alertblock}
            }
        \end{column}
        \begin{column}{0.3\textwidth}
            \vspace{-7em}
            \begin{figure}
                \centering
                \includegraphics[width=0.98\textwidth]{Figures/acgi_seasonal_turbo.png}
            \end{figure}
        \end{column}
    \end{columns}
\end{frame}

%=========================================================
\begin{frame}
    \frametitle{Introduction}
    \onslide<1->{
        \begin{columns}
                \begin{column}{0.6\textwidth}
                    \vspace{-2em}
                    \begin{figure}
                        \centering
                        \includegraphics[height=3.8cm]{Figures/tcs_cygp_gpi_interannual.png} 
                    \end{figure}
                \end{column}
                \begin{column}{0.4\textwidth}
                   \scriptsize
                   \vspace{-2.5em}
                   \begin{table}[h]
                       \centering
                       %\caption{\footnotesize Corrélation de la variabilité\\ interannuelle ERA5 avec IBTrACS}
                       \caption{\centering\footnotesize Corrélation variabilité interannuelle \mbox{entre ERA5 et IBTrACS}}
                       \begin{tabular}{l|c|c|c}
                           & TCS & CYGP & GPI \\
                           \hline
                           Ouest Pac. & 0,03 & 0,46 & 0,06 \\
                           Est Pac. & 0,7 & 0,62 & 0,64 \\
                           Nord Atl. & 0,71 & 0,7 & 0,79 \\
                           Sud Pac. & 0,35 & 0,57 & 0,39 \\
                           S.O Ind. & 0,21 & 0,37 & 0,27 \\
                           S.E Ind. & 0,22 & 0,16 & 0,12 \\
                           \hline
                           Global & 0,00 & 0,1 & 0,17
                       \end{tabular}
                   \end{table}
                \end{column}
        \end{columns}
    }
    %
    \scriptsize
    \onslide<2->{
        \begin{block}[Indice TCS (Tippet{,} Camargo et Sobel 2011)]
            \begin{equation*}
                \text{TCS} = \exp \Big( b_0 + \underbrace{\textcolor{violet}{b_\eta \eta + b_{V_{\text{shear}}} V_{\text{shear}}}}_{\text{Dynamique}} +
                \underbrace{\vphantom{b_\eta \eta + b_{V_{\text{shear}}} V_{\text{shear}}} \textcolor{red}{b_H H + b_T T}}_{\text{Thermique}} + \log \cos \phi \Big)
                %\qquad \scriptsize\text{[TC par unité de surface et par unité de temps]}
            \end{equation*}
            %
            \scriptsize
            Avec:
            \vspace{-1em}
            \begin{columns}[t]
                \begin{column}{0.5\textwidth}
                    \begin{itemize}
                        \item $\eta$ : $\min(\zeta, 3.7)$, vorticité absolue à 850~hPa \mbox{($10^{5}$ s$^{-1}$)}
                        \item $V_\text{shear}$ : Cisaillement vertical entre 850~hPa et 200~hPa (m.s$^{-1}$)
                    \end{itemize}
                \end{column}
                \begin{column}{0.5\textwidth}
                   \begin{itemize}
                        \item $H$ : Humidité relative à 600~hPa (\%)
                        \item $T$ : Écart de SST aux tropiques ($\text{SST} - \overline{\text{SST}}\big\rvert^{\tiny 20\text{°N}}_{\tiny 20\text{°S}}$) (K)
                   \end{itemize} 
                \end{column}
            \end{columns}
            %
            \vspace{0.5em}
            \scriptsize
            \begin{center}
                $\longrightarrow$ Coefficients $b$ issus d'une \underline{régression statistique} : indice facilement \alert{manipulable}
                \vspace{-1em}
            \end{center}
        \end{block}
    }
\end{frame}

%=========================================================
\subsection[Méthodologie]{Méthodologie de construction de l'indice}
\begin{frame}[c]
    \frametitle{Indices de cyclogénèse}
    \framesubtitle{Méthodologie de la régression de Poisson}
    \footnotesize
    \begin{definition}[Régression de Poisson]
        Distribution de Poisson utilisée pour modéliser des données discrètes de comptage :
        \vspace{1em}
        \begin{align*}
            \log y &= \mathbf{b}^{\text{T}} \mathbf{x} + \log (\cos \phi)\\
            \log \begin{pNiceMatrix} 
                y_1 \\
                \Vdots \\
                \\
                \\
                y_m
            \end{pNiceMatrix}
                   &=
            \begin{pNiceMatrix}
                b_1 \\
                \Vdots \\
                b_n
            \end{pNiceMatrix}
            %
            \begin{pNiceMatrix}
                x_{1,1}  & \Cdots  & x_{1,n}\\
                \Vdots & & \\
                       & & \\
                       & & \\
                x_{m,1} & & x_{m,n} \\
            \end{pNiceMatrix} + \log (\cos \phi)
        \end{align*}
        %
        \begin{itemize}
            \item Prédictant y : Densité de cyclogénèse issues des trajectoire, de taille \alert{$m = N_{\text{lat}} \times N_{\text{lon}} \times N_{\text{temps}}$}
            \item Matrice $\mathbf{x}$ : \alert{$n$} variables (colonnes) thermiques et dynamiques de grande échelle, chacune de taille $m$
            %\item Coefficients $b$ associés à chaque prédicteur (humidité relative, SST, cisaillement...)\\
            %    $\longrightarrow$ S'interprètent comme des \alert{sensibilités} !
        \end{itemize} 
    \end{definition}
    \vspace{1em}
    \begin{center}
        \small
        Indice de cyclogénèse par régression de Poisson : Choix des \underline{\alert{prédicteurs}} et \alert{coefficients} $b$
    \end{center}
\end{frame}

%=========================================================
%\subsection{Coefficients du modèle}
%\begin{frame}[c]
%    \frametitle{Indice par régression de Poisson}
%    \framesubtitle{Ré-évaluation des coefficients du TCS}
%    \small
%    \begin{definition}
%        Coefficients du TCS issus de champs \alert{climatologiques} ($N_\text{temps} = 12$) et à une résolution de \mbox{$\sim$ 2,5°}\\
%        $\longrightarrow$ \underline{Déséquilibre} de l'information spatiale et temporelle
%    \end{definition}
%    %\vspace{1em}
%    %\onslide<2->{
%    %    \begin{examples}[Hypothèse de travail]
%    %        \centering
%    %        Introduire plus de \underline{variabilité temporelle} dans la régression pour améliorer la variabilité interannuelle 
%    %    \end{examples}
%    %}
%    \vspace{2em}
%    \begin{block}[Indices construits]
%        Calcul des coefficients du TCS ($b_\eta$, $b_{V_\text{shear}}$, $b_H$, $b_T$) sur ERA5 (1980 -- 2019) avec IBTrACS comme prédictant :
%        \begin{itemize}
%            \item Champs \alert{climatologiques} (2,5° / 1,0°) : \textbf{C25} / \textbf{C10}
%            \item Champs \alert{mensuels} ($N_\text{temps} = 480$) (2,5° / 1,0°) : \textbf{M25} / \textbf{M10}
%        \end{itemize}
%        \tcblower
%        Les coefficients obtenus sont ensuite \underline{appliqués} à ERA5, pas de temps \alert{mensuel} et à \alert{1,0°} de résolution
%    \end{block}
%\end{frame}
%
%
%%=========================================================
%\begin{frame}
%    \frametitle{Indice par régression de Poisson}
%    \framesubtitle{Coefficients du modèle : Résolution spatiale et temporelle}
%    \footnotesize
%    \begin{columns}
%        \begin{column}{0.35\textwidth}
%            %\vspace{-1em}
%            \onslide<1->{
%                \begin{alertblock}
%                    \centering
%                    Pas d'effet significatif sur la variabilité interannuelle
%                \end{alertblock}
%            }
%            \onslide<2->{
%                \begin{examples}[Apport de la résolution temporelle]
%                    \setlength{\leftmargini}{2.5ex}
%                    % \begin{enumerate}
%                    %     \item Moyenne de (M25 - C25) et de \mbox{(M10 - C10)}
%                    %     \item Moyenne annuelle
%                    % \end{enumerate}
%                    Passage de coefficients calculés sur la climatologie à des coefficients calculés sur tous les mois
%                    %Effet du passage de coefficients calculés sur la climatologiques à des coefficients calculés sur tous les mois
%                \end{examples}
%            }
%            \onslide<3->{
%                \begin{block}[Analyse]
%                    \setlength{\leftmargini}{2.5ex}
%                    \begin{itemize}
%                        \item<2-> Correction du biais à l'équateur avec les indices mensuels \mbox{($b_\eta$ augmenté de 30~\%)}
%                        \item<3-> Effet de la résolution spatiale similaire mais plus faible d'un ordre de grandeur
%                    \end{itemize}
%                \end{block}
%            }
%        \end{column}
%        \begin{column}{0.65\textwidth}
%            \vspace{-1em}
%            \onslide<2->{
%                \begin{figure}
%                    \raggedleft
%                    \includegraphics[width=\textwidth]{Figures/apport_mensuel.png}
%                \end{figure}
%            }
%            %
%            \vspace{-0.8em}
%            \onslide<3->{
%                \begin{figure}
%                    \centering
%                    \includegraphics[height=3.5cm]{Figures/zonal.png}
%                \end{figure}
%            }
%        \end{column}
%    \end{columns}
%\end{frame}

%=========================================================
\subsection{Prédicteurs utilisés dans la régression}
\begin{frame}[c]
    \frametitle{Ajout de prédicteurs dans la régression}
    \small
    \onslide<1->{
        \begin{examples}[Méthodologie]
            Régressions réalisées sur les champs \alert{mensuels} à \alert{1,0°} de résolution
            %$\longrightarrow$ On ne peut pas espérer améliorer la variabilité interannuelle en ajustant sur la climatologie
        \end{examples}
    }
    \vspace{2em}
    \onslide<2->{
        \begin{block}[Prédicteurs utilisés]
            \begin{enumerate}
                \item<2-> Diagnostic du mode de variabilité climatique El Niño (ENSO):
                    \begin{itemize}
                        \item Modulateur de l'activité cyclonique à l'échelle interannuelle
                        \item Effet \underline{local} de l'ENSO présent dans les indices \parencite{watterson_seasonal_1995}, on introduit l'effet
                            \underline{distant}
                    \end{itemize}
                \item<3-> Déficit de saturation d'humidité en remplacement de l'humidité relative à 600~hPa
            \end{enumerate}
        \end{block}
    }
\end{frame}

%=========================================================
\begin{frame}[t]
    \frametitle{Ajout d'un diagnostic El Niño dans la régression}
    \scriptsize
    \begin{columns}
        \begin{column}{0.62\textwidth}
            \vspace{-3em} 
            \begin{figure}[htpb]
                \centering
                \includegraphics[height=3cm]{Figures/ONI.png}
            \end{figure} 
        \end{column}
        \begin{column}{0.38\textwidth}
            \begin{definition}[Simulation historique ARPEGE]
                \setlength{\leftmargini}{2.5ex}
                \begin{itemize}
                    \item Modèle ARPEGE v6.3 (50~km) \parencite{voldoire_evaluation_2019}
                    \item Période 1979 -- 2010
                    \item Forçage HadISST1
                \end{itemize}
                $\longrightarrow$ SST \alert{historique} mais cyclones \alert{fictifs}
            \end{definition}
        \end{column}
    \end{columns}
    %
    \vspace{2em}
    \onslide<2->{
        \begin{columns}[t]
            \begin{column}{0.6\textwidth}
                \vspace{-2em}%
                \begin{figure}
                    \includegraphics[height=4.2cm]{Figures/biais_arpege_correlation_oni.png}
                \end{figure}
            \end{column}
            \begin{column}{0.4\textwidth}
                %\vspace{1em}
                \begin{examples}[Méthodologie]
                    \setlength{\leftmargini}{2.5ex}
                    \begin{itemize}
                        \item Trajectoires détectées dans la simulation \parencite{chauvin_future_2020} comme prédictant
                        \item Prédicteurs du TCS ($\eta$, $V_\text{shear}$, $H$, $T$) + ONI \alert{répliqué} en tout point
                        \item Régression faite \alert{localement}, à l'échelle du bassin (LOA, Local / ONI / ARPEGE)
                    \end{itemize}
                \end{examples}
            \end{column}
        \end{columns}
    }
\end{frame}


%=========================================================
\begin{frame}[t]
    \frametitle{Introduction d'un diagnostic El Niño dans la régression}
    \framesubtitle{Résultats}
    \vspace{-1em}
    \begin{columns}
        \begin{column}{0.8\textwidth}
            \begin{figure}
                \centering
                \includegraphics[width=\textwidth]{Figures/apport_ONI.png}
            \end{figure}
        \end{column}
        \begin{column}{0.2\textwidth}
            \scriptsize
            \begin{examples}[Séries représentées]
                \setlength{\leftmargini}{2.5ex}
                \begin{itemize}
                    \item \underline{LOA} :\\Indice \alert{local} avec ONI
                    \item \underline{AM10} :\\Coefficients du TCS recalculés sur ARPEGE (Mensuel à 1,0°)
                    \item \underline{TCS} :\\Coefficients de \cite{tippett_poisson_2011}
                    \item \underline{Tracks} :\\Détection et suivi
                \end{itemize}
            \end{examples}
            %
            %\begin{block}
            %    \setlength{\leftmargini}{2.5ex}
            %    \begin{itemize}
            %        \item Ouest Pac. et Sud Pac. Significatifs avec ONI 
            %        \item Prédicteur souvent non significatif dans la régression
            %    \end{itemize}
            %\end{block}
        \end{column}
    \end{columns}
    %
    \onslide<2->{
        \scriptsize
        \begin{center}
            \begin{minipage}{9cm}
                \begin{alertblock}
                    \centering
                    On atteint les limites de ce que ces indices peuvent faire pour la \underline{variabilité interannuelle}
                \end{alertblock}
            \end{minipage}
        \end{center}
    }
\end{frame}

%=========================================================
\begin{frame}
    \frametitle{Déficit de saturation d'humidité}
    \framesubtitle{Écart à la saturation en remplacement de l'humidité relative à 600~hPa}
    \small
    \begin{definition}[Déficit de saturation d'humidité (VPD{,} \textit{Vapour-Pressure Deficit})]
        \footnotesize
        Motivé par des résultats encourageants en application future \parencite{camargo_testing_2014}:\\
        \[ \text{VPD} = e_s - e \]
        Avec :
        \setlength{\leftmargini}{2.5ex}
        \begin{itemize}
            \item $e_s$ : pression de vapeur saturante
            \item $e$ : Pression partielle de la vapeur d'eau
        \end{itemize}
        \vspace{\baselineskip}
        $\longrightarrow$ Le plus l'air est proche de la saturation, le plus facilement la vapeur d'eau peut condenser
        %$\longrightarrow$ \alert{Augmente} avec le \underline{réchauffement climatique}, tandis que l'humidité relative est projetée constante.\\
    \end{definition}
    \vspace{1em}
    \small
    \begin{examples}[Méthodologie]
        \footnotesize
        \begin{itemize}
            \item VPD intégré sur la colonne atmosphérique \parencite{camargo_testing_2014}
            \item Indice construit sur ARPEGE \alert{et} sur ERA5 ($y \coloneqq$ IBTrACS) : AVPD / EVPD
            \item Recherche de tendances historiques
        \end{itemize}
    \end{examples}
\end{frame}


%=========================================================
\begin{frame}[t]
    \frametitle{Déficit de saturation d'humidité}
    %\framesubtitle{Résultats}
    \begin{columns}
        \begin{column}{0.7\textwidth}
            \vspace{-1.5em}
            \begin{figure}
                \centering
                \includegraphics[height=7cm]{Figures/trends.png}
            \end{figure}
        \end{column}
        \begin{column}{0.3\textwidth}
            \scriptsize
            \vspace{-1em}
            \begin{block}[Références]
                \setlength{\leftmargini}{2.5ex}
                \begin{itemize}
                   \item \underline{Tracks} (ARPEGE) :\\
                       0,13 TC par 10 ans (p = 0.9) 
                   \item \underline{IBTrACS} (ERA5) :\\
                        -0,53 TC par 10 ans (p = 0.76)
                \end{itemize}
            \end{block}
            \vspace{1em}
            \begin{examples}[Indices]
                \setlength{\leftmargini}{2.5ex}
                \begin{itemize}
                    \item \underline{TCS} : Coefficients de \cite{tippett_poisson_2011}
                    \item \underline{AM10/EM10} : Coefficients du TCS recalculés sur ARPEGE/ERA5
                    \item \underline{AVPD/EVPD} : VPD en remplacement de $H_\text{600~hPa}$ (ARPEGE/ERA5) 
                \end{itemize}
            \end{examples}
            \vspace{1em}
            \onslide<2->{
                \begin{alertblock}
                    \centering
                    Tendances à la hausse corrigées avec le VPD
                \end{alertblock}
            }
        \end{column}
    \end{columns}
\end{frame}

%=========================================================
%\begin{frame}[c]
%    \frametitle{Indices de cyclogénèse : Vers une meilleure variabilité historique}
%    \framesubtitle{Synthèse}
%    \begin{block}[Synthèse]
%        \small
%        \begin{enumerate}
%            \setlength\itemsep{1em}
%            \item<1-> \alert{Régression de Poisson} comme méthode \textquote{objective} de construction d'indices de cyclogénèse\\
%                \vspace{0.8ex}
%                $\longrightarrow$ Permet de re-calculer les coefficients et d'expérimenter l'ajout de nouveaux prédicteurs
%            \item<2-> Variabilité interannuelle \alert{difficile} à améliorer :
%                \begin{itemize}
%                    \item Absence d'amélioration en ajoutant de la variabilité temporelle à plus haute fréquence
%                    \item Amélioration statistiquement \alert{significative} mais \textcolor{red}{modeste} avec le prédicteur ENSO 
%                \end{itemize}
%            \item<3-> Mais néanmoins :
%                \begin{itemize}
%                    \item Correction du biais à l'équateur permis par l'utilisation de champs mensuels dans la régression
%                    \item Correction de la tendance sur la période historique lorsque le déficit de saturation est utilisé comme prédicteur d'humidité
%                \end{itemize}
%        \end{enumerate}
%    \end{block}
%\end{frame}

\begin{frame}[c]
    \frametitle{Indices de cyclogénèse : Vers une meilleure variabilité historique}
    \framesubtitle{Synthèse}
    \begin{block}[Synthèse]
        \small
        \begin{enumerate}
            \setlength\itemsep{1.5em}
            \item<1-> \alert{Régression de Poisson} comme méthode \textquote{objective} de construction d'indices de cyclogénèse\\
                \vspace{0.8ex}
                $\longrightarrow$ Permet de re-calculer les coefficients et d'expérimenter l'ajout de nouveaux prédicteurs
            \item<2-> Amélioration statistiquement \alert{significative} mais \textcolor{red}{modeste} avec le prédicteur ENSO 
            \item<3-> \alert{Correction} de la tendance sur la période historique lorsque le déficit de saturation est utilisé comme prédicteur d'humidité
        \end{enumerate}
    \end{block}
\end{frame}

%=========================================================
\section{Conclusion et perspectives}
\makesecslide
\begin{frame}[c]
    \frametitle{Conclusion}
    \framesubtitle{Détection et suivi des cyclones tropicaux}
    \onslide<1->{
        \large \textbf{\textcolor{blue}{Travail sur le schéma de détection du CNRM}}\\
        \vspace{1ex}
        \small
        \begin{itemize}
            \setlength{\itemsep}{1ex}
            \item Optimisation des paramètres du traqueur par analyse de sensibilité
            \item Ajout de filtres des systèmes de moyennes latitudes en post-traitement
            \item Participation à une intercomparaison de traqueurs sur ERA5 \parencite{bourdin_intercomparison_2022}
            \item Évaluation de la représentation des cyclones dans ERA5 \parencite{dulac_assessing_2023}
        \end{itemize}
        \vspace{3em}
    }
    %
    \onslide<2->{
        \large \textbf{\textcolor{blue}{Métriques d'évaluation des performances}}\\
        \vspace{1ex}
        \small
        \begin{itemize}
            \item Développement de métriques basées sur la similarité angulaire des trajectoires détectées
        \end{itemize}
        %\begin{itemize}
        %    \item Développement de métriques d'évaluation des performances basées sur la similarité angulaire des trajectoires détectées\\
        %        \vspace{1ex}
        %        \begin{itemize}
        %            \setlength{\itemsep}{1ex}
        %            \item Métriques basées sur la décomposition d'une trajectoire en série de vecteurs de déplacements relatifs 
        %            \item Similarité intégrée sur l'ensemble de la trajectoire, par opposition au FAR et POD
        %        \end{itemize}
        %\end{itemize}
    }
\end{frame}

%=========================================================
\begin{frame}[c]
    \frametitle{Conclusion}
    \framesubtitle{Indices de cyclogénèse : variabilité historique}
    
    \onslide<1->{
        \large \textbf{\textcolor{blue}{Variabilité historique}}\\
        \small
        \vspace{1ex} 
        \begin{itemize}
            \setlength{\itemsep}{1ex}
            \item Légère amélioration lorsque le prédicteur ENSO est ajouté, au détriment du sens physique de l'indice
            \item Possibilité d'amélioration de la tendance de l'évolution de l'activité, indépendamment de la variabilité interannuelle
        \end{itemize}
    }
    \vspace{3em}
    \onslide<2->{
        \large \textbf{\textcolor{blue}{Que peut-on attendre des indices de cyclogénèse ?}}\\
        \small
        \vspace{1ex}
        \begin{itemize}
            \setlength{\itemsep}{1ex}
            \item Doit-on s'attendre à ce que les indices simulent fidèlement la variabilité interannuelle ?
            \item Les indices ont-ils encore de l'intérêt face à l'augmentation des résolutions des modèles de climat  ?
        \end{itemize}
    }
\end{frame}

%=========================================================
\begin{frame}[c]
    \frametitle{Perspectives}
    \framesubtitle{Détection et suivi des cyclones tropicaux}
    %\framesubtitle{Perspectives}
    \begin{block}[Détection et suivi]
        \small
        \begin{itemize}
            \setlength{\itemsep}{2ex}
            \item<1-> Poursuivre la recherche de l'amélioration des performances des traqueurs sur des réanalyses, en vue de leur utilisation sur des climatiques
                climatiques
                \begin{itemize}
                    \item Efficacité de détection et filtrage efficace des systèmes indésirables
                    \item Bonne capture de la cyclogénèse et de la cyclolyse
                \end{itemize}
            \item<2-> Confrontation des traqueurs \textquote{physiques} à des traqueurs IA
                \begin{itemize}
                    \item Constitution d'un objectif atteignable pour l'amélioration des traqueurs
                \end{itemize}
            \item<3-> Constitution de bases de données \textquote{expertes} de trajectoires sur les ensembles multi-modèles 
                \begin{itemize}
                    \item Ensembles \textquote{multi-traqueurs}
                    \item Constitution de bases de données composites
                \end{itemize}
        \end{itemize}
    \end{block}
\end{frame}

%=========================================================
\begin{frame}[c]
    \frametitle{Perspectives}
    \framesubtitle{Indices de cyclogénèse}
    \begin{block}[Indices de cyclogénèse]
        \small
        \begin{itemize}
            \setlength{\itemsep}{2ex}
            \item<1-> Application plus large d'indices avec le VPD sur des simulations futures, notamment sur CMIP6
            \item<2-> Utilisation de méthodologies objectives de sélection des prédicteurs
                \begin{itemize}
                    \item \cite{wang_dynamic_2020,murakami_patterns_2022} : Indice de cyclogénèse purement dynamique
                    \item Choix des prédicteurs à l'échelle des bassins
                \end{itemize}
            \item<3-> Diversification / spécilisation des indices de cyclogénèse
                \begin{itemize}
                    \item Indices régionaux, à visées opérationnelles, à différentes échelles de prévision
                    \item Indices spécialisés sur certains processus, ou caractéristique de l'activité cyclonique
                    \item Indices globaux pour une utilisation en changement climatique (si pertinent)
                \end{itemize}
        \end{itemize}
    \end{block}
\end{frame}

%=========================================================
\makethankyouslide

%\section*{Bibliographie}
%
%\begin{frame}[t,allowframebreaks]{Bibliographie}
%    \printbibliography
%\end{frame}

%=========================================================
\appendix
\section*{Annexes}
\subsection*{Optimisation du schéma de détection}

\begin{frame}[c]
    \frametitle{Optimisation des paramètres de détection}
    \framesubtitle{Étude de sensibilité du traqueur à ses paramètres}
    \begin{figure}
        \centering
        \includegraphics[width=\textwidth]{Figures/optimisation_vectors.png}
        \caption{\small Sensibilité des paramètres du traqueur en termes de FAR et POD sur les bassins NAtl et NInd entre 2008 et 2018. Chaque point correspond
        à une combinaison de 5 paramètres (243 au total).}
    \end{figure}
\end{frame}

%=========================================================
\subsection*{Représentation des TC dans ERA5}
\begin{frame}[t]
    \frametitle{Classes d'intensité croisées}
    \framesubtitle{Échelle globale}
    \begin{columns}
        \begin{column}{0.5\textwidth}
            \vspace{1em}
            \begin{figure}
                \centering
                \includegraphics[width=\textwidth]{Figures/crosstable_global_myVTU.png}
            \end{figure}
        \end{column}
        \begin{column}{0.5\textwidth}
            \footnotesize
            \setlength{\leftmargini}{3.5ex}
            \begin{examples}[Méthodologie]
                \begin{itemize}
                    \item Classes Saffir-Simpson basées sur la \alert{pression}\\\parencite{klotzbach_surface_2020}
                    \item Paires de systèmes avec moins de 75\% de valeurs manquantes
                    \item Toutes les régions sauf Nord Indien
                \end{itemize}    
            \end{examples}
            \vspace{2em}
            \begin{block}
               \begin{itemize}
                    \item Accord relatif dans l'ensemble (pente attendue)
                    \item \alert{Dispersion importante} vers les catégories les plus hautes
               \end{itemize} 
            \end{block}
        \end{column}
    \end{columns} 
\end{frame}

%=========================================================
\begin{frame}[t]
    \frametitle{Classes d'intensité croisées}
    \framesubtitle{Contingence par bassin océanique}
    \begin{columns}
        \begin{column}{0.85\textwidth}
            \begin{figure}
                \centering
                \includegraphics[width=\textwidth]{Figures/Annexes/crosstable_region.png}
            \end{figure}
        \end{column}
        \begin{column}{0.15\textwidth}
            \scriptsize
            \begin{block}
               Relation et dispersion variable selon les régions 
            \end{block}
        \end{column}
    \end{columns}
\end{frame}

%=========================================================
\begin{frame}[t]
    \frametitle{Composites de la structure interne des TC dans ERA5}
    \begin{figure}
        \centering
        \includegraphics[width=\textwidth]{Figures/Annexes/all_composites.png}
        \caption{\small Composites ERA5 moyennés par classe d'intensité de la composante IBTrACS, indépendamment de l'intensité dans ERA5.}
    \end{figure}
\end{frame}

%=========================================================
\subsection*{Trajectoires simulation ARPEGE historique}
\begin{frame}[c]
    \frametitle{Densité de cyclogénèses détectées dans la simulation ARPEGE historique}
    \begin{figure}
        \centering
        \includegraphics[width=0.8\textwidth]{Figures/Annexes/track_density_PRE625REFT359x.png}
        \caption{\small Densité annuelle moyenne de cyclogénèses dans la simulation ARPEGE forcée par HadISST1 entre 1980 et 2010, calculée sur une grille régulière de 2°×2°.}
    \end{figure}
    
\end{frame}

\subsection*{Ajout de l'ONI dans la régression de Poisson}
\begin{frame}[c]
    \frametitle{Corrélation cyclogénèses détectées avec l'ONI}
    \framesubtitle{Par point de grille}
    \begin{figure}
        \centering
        \includegraphics[width=0.8\textwidth]{Figures/Annexes/corr_ONI_tracks.png}
        \caption{\small Carte de corrélation entre la variabilité interannuelle de l’activité cyclonique par point de grille (3 °x3 °) pour les cyclogénèses détectées dans ARPEGE et l’ONI, sur les saisons cycloniques entre 1980 et 2010.}
    \end{figure}
\end{frame}

\end{document}
